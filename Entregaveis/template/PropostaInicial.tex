%% o documento é definido do tipo "book" em a4, font 12pt
\documentclass[a4paper,12pt]{book}

%% para usar caracteres acentuados
\usepackage[utf8]{inputenc} % no Unix (com codificação UTF-8)
%\usepackage[latin1]{inputenc} % no Windows (com codificação ISO-8859-1, ou ISO-8859-15)

% para aspectos de hifenização (do Português)
\usepackage[portuguese]{babel}

%% para incluir imagens e tratar de modo adequado endereços url
\usepackage{graphicx,url}

%% para usar \begin{comment} ... \end{comment}
\usepackage{verbatim}

%% para usar o símbolo do euro
%usepackage{eurosym}

%% para juntar multiplas linhas em tabelas
\usepackage{multirow}

%% Para usar símbolos matemáticos
\usepackage[centertags]{amsmath}

%% Para usar \mathds{...}
\usepackage{dsfont}

%% para formatação (e.g. espaçamento) de Tabelas
\usepackage{tabls}

%% Para usar listagens de código
\usepackage{listings}

%% para escrita "formal" de algoritmos
\usepackage{algorithm}

%% Para que a numeracao de listings seja global
%% (e nao no contexto de cada capitulo!)
\usepackage{chngcntr}

% Para usar um estilo diferente na identificação de "Capítulo"
%Sonny, Lenny(x), Glenn, Conny, Rejne, Bjarne
%\usepackage[Lenny]{fncychap}
% Para ter no "heading" o nome do capítulo e da secção
\usepackage{fancyhdr}

%%________________________________________________________________________
%% para incluir o estilo proposto
\usepackage{./PRJ_Style}
%%________________________________________________________________________
%%________________________________________________________________________
% mudar alguns nomes fixos para Português
%%________________________________________________________________________
% renomear "Listing" para "Código"
\renewcommand{\lstlistingname}{Código}
\addto\captionsportuges{
    \renewcommand{\contentsname}{Índice}}
\addto\captionsportuges{
    \renewcommand{\bibname}{Bibliografia}}
\addto\captionsportuges{
    \renewcommand{\proofname}{Prova}}
\addto\captionsportuges{
    \renewcommand{\chaptername}{Capítulo}}
\setlength{\headheight}{15pt}
%%________________________________________________________________________





%%________________________________________________________________________
%%%%%%%%%%%%%%%%%%%%%%%%%%%%%%%%%%%%%%%%%%%%%%%%%%%%%%%%%%%%%%%%%%%%%%%%%%
%% Begin Document
%%%%%%%%%%%%%%%%%%%%%%%%%%%%%%%%%%%%%%%%%%%%%%%%%%%%%%%%%%%%%%%%%%%%%%%%%%
\begin{document}
% para que a numeracao de listings seja global
% (e nao no contexto de cada capitulo!)
\counterwithout{lstlisting}{chapter}
%% incluir a capa
\frontmatter
\fazerCapa
%% incluir o resumo e abstract
%% caso pretenda, incluir os agradecimentos e a dedicatória
\frontmatter
%%________________________________________________________________________
%% comentar o que não interessar
%%________________________________________________________________________
\myPrefaceChapter{Resumo}
%%________________________________________________________________________

Lorem ipsum dolor sit amet, consectetur adipiscing elit. Curabitur convallis purus quis lacinia convallis. Duis sit amet lectus sed ex laoreet mattis in ac lacus. Ut nec semper purus. Cras elementum rhoncus semper. Ut luctus nisl et magna rhoncus, non imperdiet nisi convallis. Praesent aliquet dapibus quam vel fringilla. Quisque diam lacus, posuere eu ultrices id, sollicitudin ac orci. Pellentesque ac sollicitudin lacus, id imperdiet nisi. Phasellus eleifend nunc augue, vel molestie justo tristique et. Ut fermentum placerat metus ac cursus. Etiam malesuada maximus convallis. Nam sagittis ex eget nibh tristique, sodales posuere felis ornare. Nullam eget felis sed sem aliquet vestibulum id non nunc. Mauris ante quam, venenatis eu ante vitae, commodo cursus arcu.

%%________________________________________________________________________
\myPrefaceChapter{Abstract}
%%________________________________________________________________________

Pellentesque a nisi consequat, laoreet eros at, accumsan sapien. Nulla facilisi. Maecenas fermentum malesuada orci nec vestibulum. Vestibulum ante ipsum primis in faucibus orci luctus et ultrices posuere cubilia Curae; Nullam quis ex sem. Morbi id commodo enim, sit amet cursus ipsum. Duis imperdiet sed elit convallis cursus. Cras ut orci velit. Vivamus commodo sagittis libero, venenatis semper ipsum vestibulum finibus. Sed at ipsum orci.
%%________________________________________________________________________
\myPrefaceChapter{Agradecimentos}
%%________________________________________________________________________

Pellentesque a nisi consequat, laoreet eros at, accumsan sapien. Nulla facilisi. Maecenas fermentum malesuada orci nec vestibulum. Vestibulum ante ipsum primis in faucibus orci luctus et ultrices posuere cubilia Curae; Nullam quis ex sem. Morbi id commodo enim, sit amet cursus ipsum. Duis imperdiet sed elit convallis cursus. Cras ut orci velit. Vivamus commodo sagittis libero, venenatis semper ipsum vestibulum finibus. Sed at ipsum orci.
\paragraph{2}
lorem impsum

%%________________________________________________________________________
\begin{myDedication}
%%________________________________________________________________________

Mauris fermentum massa eget libero vulputate, nec consequat ligula placerat. Duis varius, lorem ut volutpat fermentum, massa ligula vehicula nisl, a porttitor augue libero eget dolor. Vestibulum ante ipsum primis in faucibus orci luctus et ultrices posuere cubilia Curae; Nullam bibendum urna massa, sit amet scelerisque mi euismod id. Nulla euismod imperdiet tellus, sed pretium neque pharetra ac. Nulla sollicitudin nulla quis varius convallis. Pellentesque habitant morbi tristique senectus et netus et malesuada fames ac turpis egestas. Proin consequat leo lacus, et egestas ligula mollis eget.
    
\end{myDedication}
%%________________________________________________________________________





%% incluir a lista de conteúdos (índice, tabelas e figuras)
%comentar se quiser alterar o espaçamento entre linhas
\setlinespacing{1.15} 
\myTableOfContents
%%________________________________________________________________________
%% comentar o que não interessar
\myListOfTables
\myListOfFigures
%%________________________________________________________________________





%% "Corpo Principal" do texto
\mainmatter
\myPageStyle
%% cada um dos Capítulos
%% (aqui separados em dois ficheiros)
%%________________________________________________________________________
%% comentar o que não interessar (ou incluir outros ficheiros)
\input{./PRJ_intro.tex}
%%________________________________________________________________________




\appendix
%%________________________________________________________________________
%% comentar o que não interessar (ou incluir outros ficheiros)
%\input{./02_PRJ_cap_apendice.tex}
%%________________________________________________________________________





%% bibliografia
\begin{thebibliography}{1}
\bibitem{a} 

\bibitem{b} 

\bibitem{c} 

\end{thebibliography}

\chapter*{Webografia}
\begin{itemize}
	\item[1]  

	\item[2] 
    
	\item[3] 

	\item[4] 

	\item[5] 

	\item[6] 

	\item[7] 
\end{itemize}

%%________________________________________________________________________
\end{document}