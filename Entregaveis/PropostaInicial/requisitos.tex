\renewcommand{\labelitemii}{$\star$}
After analysing the objectives of this project we identified functional and non-functional requirements.
\\
\\
Functional:
\begin{description}[font=$\bullet$~\normalfont\scshape\color{red!50!black}]
\item \underline{Multi-Language} Provide run environment for multiple programming languages.(eg. Java, Kotlin, C\#, JS, Python)

\item \underline{Execute Solution} Any user may write a code solution, run it and get the result trough the front-end application.

\item \underline {Basic Authentication} Users can create an account, with a profile information and will be able to get access to more features(eg. keep track of the solved challenges, create/solve Questionnaires to/from other users).

\item \underline{Challenges} are a programming problem that needs to be solve. Every challenge has a built in answer that will be compared with the user submitted solution to determinate its "correctness" trough unit tests.
   \begin{itemize}
     \item Any user may create a Challenge, but also has to create its unit tests, and they have to compile/run successfully before being able to submit it. Only its creator can edit the Challenge.
     \item Challenges may have tags associated, and they can be searched by them.
     \item A logged in user can track the Challenge he/she submitted.
   \end{itemize}
\item \underline{Questionnaires} are a group of Challenges.
	\begin{itemize}
	\item Questionnaires can only be submitted or solved by logged in users.
	\item Only its creator can edit the Questionnaire.
    \item A user may choose any number of Challenges to create a Questionnaire and set it a timer. 
    (eg. user A created a Questionnaire with 5 Challenges that has to be solved by user B in one hour).
    \item A user may save its favourite Questionnaires.
    \item Questionnaires can be shared through a link created by the platform.
	\end{itemize}
\end{description}

\begin{flushleft}
Non-Functional Requirements:
\end{flushleft}

\begin{description}[font=$\bullet$~\normalfont\scshape\color{red!50!black}]
\item \underline{Scalability} This platform may be accessed by hundreds or even thousands of users. As such the proposed architecture takes into consideration the scalability of its usage.
         
\item \underline{Security} Executing third party code in a machine raises security concerns.
	\begin{itemize}
		\item 
		\iffalse		
		Executing the services which will execute this code in containers adresses some of these concerns.
		\fi 
		A self-contained run environment limits the impact of malicious code to the container which executes it, protecting the remaining infrastructure.
	\end{itemize}	   
              
\item \underline{Solution Maintenance}     
        Maintaining a complex solution requires a balance between many moving parts, as such this project's architecture reflects the principles of loose coupling and modularity which facilitate the solution maintenance.      

\iffalse
\item \underline{Efficiency}    
        Hosting the solution in a cloud based environment improves efficiency of the solution. 
\fi
\item \underline{Open source}     
        Open source software refers to something people can modify and share because its design is publicly accessible.

\end{description}