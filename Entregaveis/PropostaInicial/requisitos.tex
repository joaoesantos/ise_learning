\renewcommand{\labelitemii}{$\star$}
After analysing the objectives of this project we identified functional and non-functional requirements.
\\
\\
Functional:
\begin{description}[font=$\bullet$~\normalfont\scshape\color{red!50!black}]
\item [Multi-Language] Provide run environment for multiple programming languages.(eg. Java, Kotlin, C\#, JS, Python)

\item [Execute Solution] Any user may write a code solution, run it and get the result trough the front-end application.

\item [Basic Authentication] Users can create an account, with a profile information and will be able to get access to more features(eg. keep track of the solved challenges, create/solve Questionnaires to/from other users).

\item [Challenges] are a programming problem that needs to be solve. Every challenge has a built in answer that will be compared with the user submitted solution to determinate its "correctness" trough unit tests.
   \begin{itemize}
     \item Any user may create a Challenge, BUT also has to create its unit tests, and they have to compile/run successfully before being able to submit it. Only its creator can edit the Challenge.
     \item Challenges may have tags associated, and they can be searched by them.
     \item A logged in user can track the Challenge he/she submitted.
   \end{itemize}
\item[Questionnaires] are a group of Challenges.
	\begin{itemize}
	\item Questionnaires can only be submitted or solved by logged in users, and they consist in a group of existing Challenges.
    \item A user may choose any number of Challenges to create a Questionnaire and set it a timer. Only its creator can edit the Questionnaire.
    (eg. user A created a Questionnaire with 5 Challenges that has to be solved by user B in one hour).
    \item A user may save its favourite Questionnaires.
    \item Questionnaires may be shared trough a link.
    \item Only logged in users have access to the Questionnaires.
	\end{itemize}
\end{description}

\begin{flushleft}
Non-Functional Requirements:
\end{flushleft}

\begin{description}[font=$\bullet$~\normalfont\scshape\color{red!50!black}]
\item[Scalability] This platform may be accessed by hundreds or even thousands of users. As such the proposed architecture takes into consideration the scalability of its usage.
\\
	\begin{itemize}
		\item The microservice architecture allows tasks to be broken down into smaller units since the services follow the single responsibility principle.
		\item Hosting these services in containers and managed in a Kubernetes cloud based environment makes it easier to scale up instances of those services, meeting high peek demands when necessary.
	\end{itemize}
         
\item[Security] Executing third party code in a machine raises security concerns.
	\begin{itemize}
		\item Executing the services which will execute this code in containers adresses some of these concerns. A self-contained run environment limits the impact of malicious code to the container which executes it, protecting the remaining infrastructure.
	\end{itemize}	   
              
\item[Solution Maintenance]     
        Maintaining a complex solution requires a balance between many moving parts, as such this project's architecture reflects the principles of loose coupling and modularity which facilitate the solution maintenance. 
		\begin{itemize}
			\item The module separation of the platform maintains loose coupling between the modules.
			\item The microservice architecture maintains the services scope smaller and reduces dependence from services implementation.
			\item Components will be deployed in containers the infrastructure maintenance will be easier because it will only need to support Docker containers
		\end{itemize}		        

\item[Efficiency]     
        Hosting the solution in a cloud based environment improves efficiency of the solution. 
		\begin{itemize}
			\item Cloud based hosting provides load balancing for services, improving network load efficiently across multiple servers.
			\item Cloud based hosting allows for a more efficient resource management, allowing for scaling up or down resource usage adjusting to current demand.
		\end{itemize}		        
         

\item[Infrastructure Agnostic]     
        Because every component will be deployed in separate containers the deployment will be platform independent and fault tolerant to component failure when managed in a Kubernetes cloud based environment.

\item[Open source]     
        Open source software refers to something people can modify and share because its design is publicly accessible.
        \begin{itemize}
        	\item Improved control since the is publicly available.
        	\item Improved security since the code is accessible to be tested and verified stimulating error detection.
        	\item Improved stability since the source code is always available
        	\item Fosters community, group of people invested in producing, testing, using and promoting the software.
        \end{itemize}

\end{description}